
% Capitulos sao secoes
% Artigos e paragrafos sao enumerates

\documentclass{article}
\usepackage{enumitem}
\usepackage[utf8]{inputenc}
\usepackage{titlesec}

\titleformat{\section}{\normalfont\scshape}{Capítulo \Roman{section}}{1em}{}

\setlist[enumerate,1]{label=Art. \theenumi,resume}

\setlist[enumerate,2]{label=§ \arabic*$^{\circ}$.,align=left}

\newcommand{\singleitem}{\item[Parágrafo Único.]}


\title{Regimento PPGC}

\begin{document}

\maketitle

\section{Das finalidades}


\begin{enumerate}
\item O Programa de Pós-Graduação em Computação da Universidade Federal de Pelotas, neste documento referenciado por PPGC ou simplesmente por Programa, em nível de Mestrado e Doutorado, tem por finalidade a formação de recursos humanos para o ensino e pesquisa capazes de realizar projetos de investigação científica, incluindo aspectos de planejamento, delineamento, execução, análise e publicação, contribuindo com o avanço do conhecimento científico e tecnológico em Computação.
\end{enumerate}

\section{Da Administração do Programa}

\begin{enumerate}

	\item O PPGC é administrado pelo Colegiado do Programa, sendo este o órgão superior do Programa, com funções normativas, deliberativas e executivas.

	\item O Colegiado é composto pelos seguintes membros:
	\begin{enumerate}[label=\Roman*]
		\item Coordenador do Programa.
		\item Docentes do programa em número igual a quarta parte do total de docentes, arredondado para o número inteiro mais próximo;
		\item Representante Discentes, em número de um para cada curso do Programa.
	\end{enumerate}

	\item Os membros docentes do Colegiado são eleitos pelo corpo docente do Programa por meio de votação.
	\begin{enumerate}
		\item Cada Linha de Pesquisa terá direito a uma posição no Colegiado, na forma do seu membro com maior número de votos.
		\item As demais posições serão preenchidas na ordem especificada na votação, do docente com maior número de votos para o com menor número.
	\end{enumerate}

	\item O Coordenador e o Coordenador Adjunto são eleitos entre os membros do corpo docente do Programa através de votação pelo Colegiado. 
	\begin{enumerate}
		\item O Coordenador e Coordenador Adjunto serão os membros com o primeiro e segundo maior número de votos, respectivamente.
		\item O Colegiado é presidido pelo Coordenador do Programa ou, na ausência deste, pelo Coordenador Adjunto, seguido pelo membro mais antigo na Instituição pertencente ao Colegiado.
		\item O Coordenador terá mandato de dois anos e será permitida apenas uma recondução sucessiva ao cargo.
	\end{enumerate}

	\item Os Representantes Discentes são eleitos através de votação pelos alunos regulares dos respectivos cursos que representarão.
	\begin{enumerate}
		\singleitem O aluno mais votado assumirá a função de Representante Discente enquanto o segundo aluno mais votado assumirá a função de suplente do Representante Discente.
	\end{enumerate}	

	\item Assessoram nas decisões do Colegiado três Comissões permanentes, cujos membros são indicados pelo Colegiado entre os membros do corpo docente do Programa:
	\begin{enumerate}[label=\Roman*]
		\item Comissão de Acompanhamento Discente
		\item Comissão de Avaliação do Programa
		\item Comissão de Seleção
	\end{enumerate}

\end{enumerate}

\section{Das Atribuições do Colegiado}
\begin{enumerate}

	\item O Colegiado de Programa de Pós-Graduação reunir-se-á, quando convocado pelo seu Coordenador ou por, no mínimo, metade dos seus membros.
	\begin{enumerate}
		\singleitem O Colegiado do Programa só se reunirá com a presença da maioria de seus membros. 
	\end{enumerate}

	\item O Colegiado do Programa deliberará por maioria simples de votos dos membros presentes.
	\begin{enumerate}
		\item O Coordenador do Colegiado não possui direito a voto. 
		\item Todos os demais membros do Colegiado possuem direito a voto, em igual peso.
		\item O Coordenador dará o Voto de Qualidade em caso de empate na votação. % Isso eh diferente de dizer que o Coordenador tem direito a voto apenas em caso de empate?
	\end{enumerate}


	\item Compete ao Colegiado do Programa:
	\begin{enumerate}[label=\Roman*]
		\item Executar as diretrizes estabelecidas pela Pró-Reitoria de Pesquisa e Pós-Graduação e pelo Conselho Coordenador de Ensino, Pesquisa e Extensão desta Instituição;
		\item Elaborar o Regimento do Programa de Pós-Graduação contendo as normas relativas ao funcionamento do mesmo, para aprovação pela Câmara de Pós-Graduação ``Stricto Sensu'' e pelos demais órgãos competentes;		
		\item Estabelecer metas de qualidade de curto, médio e longo prazos para o Programa, bem como estratégias para atingir estas metas;
		\item Exercer a coordenação interdisciplinar, visando a conciliar os interesses de ordem didática da Unidade com o do Programa de Pós-Graduação;
		\item Elaborar e manter atualizadas as informações didáticas do Programa;
		\item Fixar a sequência recomendável de estudos e os pré-requisitos necessários;
		\item Emitir parecer sobre assuntos de interesse do Programa de Pós-Graduação;
		\item Analisar e emitir parecer sobre os pedidos de transferência, aproveitamento de estudos e adaptações, de acordo com as normas fixadas pelo Conselho Coordenador de Ensino, Pesquisa e Extensão e a regulamentação estabelecida pelo Conselho de Pós-Graduação;
		\item Julgar, em grau de recurso, decisões proferidas pelo Coordenador do Programa;
		\item Verificar o cumprimento do Conteúdo Programático e da Carga Horária das disciplinas dos cursos; % Cumprimento por quem? Docentes ou alunos?
		\item Estabelecer mecanismos de orientação acadêmica aos alunos dos cursos;
		\item Acolher, avaliar, solicitar alterações e aprovar o plano de estudo de cada aluno antes do final do primeiro período letivo; 
		\item Promover o acompanhamento dos alunos por meio de registros individuais;
		\item Homologar as dissertações e teses após a banca de defesa e após realizadas as correções exigidas pela banca examinadora, se alguma;
		\item Homologar a nominata para Banca Examinadora de cada pedido de Defesa de Dissertação ou Defesa de Tese recebido;
		\item Instalar, anualmente, uma Comissão de Seleção de Ingresso para encaminhamento do Processo Seletivo de candidatos ao ingresso no Programa.
		\item Indicar, à ocasião do Processo Seletivo de novos ingressantes, os Orientadores para cada candidato selecionado; % Não está incluso no 'Estabelecer mecanismos de orientação'?
		\item Propor aos órgãos competentes da Universidade a interrupção, suspensão ou cessação das atividades do Programa;
		\item Avaliar anualmente o desempenho global do PPGC;
		\item Realizar regularmente a avaliação do Corpo Docente promovendo o descredenciamento de membros junto a Programa quando necessário;
		\item Indicar as Comissões de assessoramento;
		\item Manifestar-se sobre as Regras de Avaliação do Programa e as Regras de Avaliação Docente propostas pela Comissão de Avaliação;
		\item Analisar e se pronunciar sobre o Relatório de Avaliação do Programa e propor ações cabíveis para melhora de sua qualidade;
		\item Receber, avaliar e apresentar julgamento sobre pedidos de credenciamento de docentes junto ao Programa;
		\item Se pronunciar sobre prioridades de aplicação de recursos específicos do Programa;
		\item Reunir-se para escolha de novo Coordenador de Programa e Coordenador Adjunto quando terminado o mandato de dois anos ou quando necessário; 

		\item Definir a Meta Mínima de Produção e a Meta Global de Produção para fins de processos de avaliação dos docentes;

		\item Estabelecer a regularidade dos processos seletivos para novos discentes, bem como os critérios gerais de seleção;

		\item Resolver, nos limites de sua competência, os casos omissos deste Regimento.
	\end{enumerate}
	\begin{enumerate}
		\singleitem Recursos às decisões do Colegiado de Programa devem ser dirigidos à Câmara de Pós-Graduação ``Stricto Sensu'' da Pró-reitoria de Pesquisa e Pós-Graduação desta Universidade.
	\end{enumerate}
\end{enumerate}

\section{Das Atribuições do Coordenador do Programa}
\begin{enumerate}
	\item Ao Coordenador de Programa, compete:
	\begin{enumerate}[label=\Roman*]
		\item Coordenar e supervisionar o funcionamento do Programa;
		\item Convocar e presidir as reuniões do Colegiado do Programa, com direito ao voto de qualidade;
		\item Representar o Colegiado e as decisões tomadas neste fórum;
		\item Enviar, semestralmente, à Pró-Reitoria de Pesquisa e Pós-Graduação, de acordo com o calendário vigente, ouvidos os professores envolvidos, a relação de disciplinas a serem ofertadas com os respectivos professores responsáveis;
		\item Enviar à Pró-Reitoria, em tempo oportuno, as necessidades de bolsas, bem como sua distribuição entre os discentes;
		\item Elaborar os relatórios anuais destinados às instituições fornecedoras de bolsas, enviando-os à Pró-Reitoria de Pesquisa e Pós-Graduação;
		\item Comunicar ao órgão competente qualquer irregularidade no funcionamento do Programa e solicitar as correções necessárias;
		\item Designar Relator ou Comissão para estudo de matéria submetida ao Colegiado;
		\item Articular o Colegiado com os Departamentos e outros órgãos envolvidos;
		\item Decidir sobre matéria de urgência ``ad referendum'' do Colegiado;
		\item Exercer outras atribuições inerentes ao cargo;
		\item Supervisionar e zelar pela aplicação das verbas específicas do Programa.
	\end{enumerate}
	\item Ao Coordenador Adjunto de Programa, compete Substituir o Coordenador em suas ausências ou impedimentos, auxiliá-lo na execução das deliberações do Colegiado e executar as tarefas que lhe forem especificamente designadas pelo Colegiado.
\end{enumerate}

\section{Das Atribuições da Comissão de Acompanhamento Discente}
\begin{enumerate}
	\item Compete à Comissão de Acompanhamento Discente:
	\begin{enumerate}[label=\Roman*]
		\item Monitorar, ao final de cada período letivo, o desempenho acadêmico dos discentes do Programa;
		\item Identificar discentes em risco de não cumprir os requisitos necessários à conclusão do curso e levar estes casos à Coordenação e aos Orientadores relevantes aos casos;
		\item Propor aos Orientadores, quando relevante, ações corretivas a discentes em risco;
		\item Identificar discentes que estejam fora das normas estabelecidas pelo Regimento e pelo Colegiado, levando estes casos à Coordenação.
	\end{enumerate}
\end{enumerate}

\section{Das Atribuições da Comissão de Avaliação do Programa}
\begin{enumerate}
	\item Compete à Comissão de Avaliação:
	\begin{enumerate}[label=\Roman*]
		\item Estabelecer métricas relevantes ao acompanhamento da qualidade do Programa, considerando regulamentos relevantes da CAPES e o Documento de Área vigente;
		\item Estabelecer valores desejáveis às métricas estabelecidas, considerando as metas de curto, médio e longo prazos definidas pelo Colegiado;
		\item Monitorar as métricas estabelecidas, com frequência não inferior a uma vez por ano;
		\item Relatar ao Colegiado os resultados do monitoramento;
		\item Propor ações para melhoria da qualidade do Programa.
	\end{enumerate}
\end{enumerate}

\section{Das Atribuições da Comissão de Seleção}
\begin{enumerate}
	\item Compete à Comissão de Seleção:
	\begin{enumerate}[label=\Roman*]
		\item Propor ao Colegiado critérios específicos de seleção de novos discente ao Programa;
		\item Elaborar o Edital de Seleção para cada processo seletivo de novos discentes ao Programa;
		\item Homologar as inscrições dos processos seletivos;
		\item Conduzir o processo seletivo e submeter os resultados ao Colegiado;
		\item Elaborar e manter documentação dos processos seletivos.
	\end{enumerate}
\end{enumerate}

\section{Do Corpo Docente}

\begin{enumerate}
	\item O Corpo Docente do PPGC é constituído por professores e pesquisadores, portadores de título de doutor, responsáveis por ministrar disciplinas regulares no Programa e habilitados a orientar e co-orientar teses e dissertações.

	\begin{enumerate}
		\item O Corpo Docente do Programa deve ser constituído, majoritariamente, por docentes da Universidade Federal de Pelotas.

		\item Poderão integrar o Corpo Docente do Programa, inclusive, como Professor Responsável de Disciplina, professores portadores de título de doutor, de outras Instituições de Ensino Superior, nacionais ou estrangeiras, de centros de pesquisa, bem como outros profissionais portadores de título de doutor, do país ou do exterior.

		\item Solicitações de credenciamento junto ao Corpo Docente do Programa podem ser encaminhadas a qualquer momento ao Colegiado, que deliberará sobre o pedido considerando a capacidade do candidato em contribuir para o Programa, além de quaisquer resoluções vigentes à época.
	\end{enumerate}

	\item Para efeito de enquadramento junto ao Programa, define-se como Docente Ativo o docente que, em um dado ano, tenha atuado no Programa realizando duas ou mais atividades naquele ano, entre: (a) lecionar uma disciplina, (b) orientar ou co-orientar uma dissertação de mestrado ou tese de doutorado, (c) participar na produção científica qualificada do Programa. Os docentes serão designados como:

	\begin{enumerate}[label=\Roman*]

		\item Permanentes – servidores da Universidade Federal de Pelotas que respondam à definição de Docente Ativo.

		\item Visitantes – identificados por possuirem vínculo com alguma instituição, no Brasil ou no Exterior, que permanecerem, durante um período contínuo e determinado, à disposição do Programa, contribuindo para o desenvolvimento das atividades acadêmico-científicas deste.		

		\item Colaboradores – demais docentes credenciados junto ao Programa.

	\end{enumerate}

	\item São atribuições dos docentes:
	\begin{enumerate}[label=\Roman*]
		\item 	Desenvolver projetos de pesquisa que contribuam para a área de Computação;
		\item	Atuar como Orientador ou Co-orientador em Dissertações de Mestrado e Teses de Doutorado de alunos do Programa;
		\item	Ministrar aulas teóricas e práticas de disciplinas do Programa, de acordo com o programa vigente de cada Disciplina;
		\item	Manter o Registro Acadêmico da Disciplina, bem como o Registro de Desempenho individual de cada aluno nela inscrito;
		\item	Atualizar e divulgar o programa da disciplina a cada edição desta;
		\item	Promover e participar de seminários, simpósios e estudos dirigidos;
		\item	Participar de Comissões Examinadoras;
		\item	Responder à Comissão de Avaliação do Programa quando solicitado;
		\item	Desenvolver pesquisa que resulte em produção científica divulgada em periódicos indexados;
		\item	Divulgar resultados de pesquisas em eventos e periódicos qualificados;
		\item	Promover integração com a região prestigiando eventos científicos regionais;
		\item	Promover a pesquisa em Computação nos cursos de graduação da área;
		\item	Acatar as decisões do Colegiado e executar as tarefas que neste fórum lhe foram atribuídas no prazo conveniado;
		\item	Desempenhar demais atividades, dentro dos dispositivos regimentais, que possam beneficiar o Programa.
	\end{enumerate}

	\begin{enumerate}
		\item É assegurada ao Docente autonomia didática, nos termos da legislação vigente, do regimento da Universidade Federal de Pelotas e deste Regimento.

		\item O docente que ao final de um ano não atingir a meta de produção mínima definida pelo Colegiado, não poderá acolher novos Orientandos;

		\item O docente que em dois anos consecutivos não atingir a meta de produção mínima definida pelo Colegiado, não terá disciplina a ele atribuída.

		% Acho que aqui deveria ser "não atingir a meta E o programa não atingir a meta global"
	\end{enumerate}
\end{enumerate}

\section{Do Credenciamento e Descredenciamento}
\begin{enumerate}
	\item Docentes serão considerados para credenciamento junto ao Programa mediante pedido por escrito ao Colegiado;

	\item O solicitante deverá atender aos requisitos mínimos definidos em resolução própria do Programa e deverá ser aprovado pelo Colegiado;

	\item Docentes credenciados serão desligados do Programa quando:
	\begin{enumerate}
		\item Durante três avaliações consecutivas, não atingir a Meta Mínima de Produção definida pelo Colegiado, exceto nos casos onde:
		\begin{enumerate}
			\item A Meta Global de Produção anual do Programa, definida pelo Colegiado, for atingida; ou,
			\item O docente possuir orientações em andamento.
		\end{enumerate}

		\item As avaliações serão realizadas anualmente, no primeiro trimestre de cada ano.
	\end{enumerate}

\end{enumerate}


\section{Da Orientação}

\begin{enumerate}
	\item Cada Aluno ingressante no Programa contará com um Orientador e deverá se reportar à Comissão de Acompanhamento Discente.
	\begin{enumerate}
		\item Compete ao Colegiado do Programa determinar o Orientador a cada Candidato selecionado para ingresso no Programa observando as informações apresentadas pelo candidato e a disponibilidade de orientação dos membros do Programa.
		\item A alteração de Orientação pode ser solicitada a qualquer tempo que anteceda seis meses a data da Defesa da Dissertação ou Tese e será objeto de apreciação e parecer do Colegiado.
	\end{enumerate}

	\item Os Professores Orientadores do PPGC são os membros do Corpo Docente deste Programa. 
	\begin{enumerate}
		\item O número máximo de orientações simultâneas em cada nível será determinada por regulamentação do Colegiado.
		\item Somente estão habilitados a orientar alunos de Doutorado os docentes do Programa que possuírem pelo menos uma orientação concluída de aluno de Mestrado ou Doutorado, como orientador principal.
	\end{enumerate}

	\item  São atribuições do Professor Orientador:
	\begin{enumerate}[label=\Roman*]
		\item Elaborar, juntamente com o aluno, o Plano de Estudos a ser desenvolvido, incluindo disciplinas a serem cumpridas e Proposta de Dissertação ou Tese, e encaminhá-lo ao Colegiado, dentro dos prazos regulamentares;
		\item	Orientar o aluno no Plano de Estudos, desde sua concepção até a redação final;
		\item	Promover o bom andamento do projeto de pesquisa aprovado pelo Colegiado, respeitando os prazos estabelecidos pelo Programa;
		\item	Atuar na captação de recursos financeiros para custear o desenvolvimento dos projetos de pesquisa de seus orientandos;
		\item	Orientar, acompanhar e assinar a matrícula dos seus orientandos a cada semestre;
		\item	Indicar ao Colegiado, se julgar conveniente, o(s) Co-orientador(es) de seus orientandos;
		\item	Autorizar os orientandos a apresentarem suas Dissertações ou Teses;
		\item	Sugerir ao Colegiado os nomes dos integrantes da Banca Examinadora e a data para a realização da defesa de seus orientandos;
		\item	Presidir a Banca Examinadora de Defesa de Dissertação ou Tese de seus orientandos.
	\end{enumerate}

	\item O papel de Co-orientador deve representar a complementação de conhecimentos envolvidos em um trabalho que envolva duas ou mais áreas ou que represente notório saber.
	\begin{enumerate}
		\item O pedido de inclusão de Co-orientador deve ser encaminhado pelo Orientador para apreciação e aprovação do Colegiado em uma data que anteceda, pelo menos, seis meses a data da Defesa do Orientando.
		\item Na impossibilidade do Orientador presidir a Banca Examinadora, compete a um dos Co-orientadores, presidi-la.
		\item É vedada a composição de uma Banca Examinadora com o Orientador e um ou mais dos eventuais Co-orientadores.
	\end{enumerate}

\end{enumerate}

\section{Da Seleção e Matrícula}
\begin{enumerate}
	\item Serão considerados para admissão ao curso de Mestrado candidatos que sejam portadores de diploma de graduação reconhecido por órgão competente.
	\item Serão considerados para admissão ao curso de Doutorado candidatos que sejam portadores de diploma de Mestre, modalidade Mestrado Acadêmico, reconhecido por órgão competente.

	\item Em caráter excepcional, a critério do Colegiado e por indicação de docente do Programa, poderão ser considerados para admissão no curso de Doutorado candidatos apenas com diploma de graduação.

	\item Em caráter excepcional, a critério do Colegiado e por requisição do Orientador, alunos matriculados no curso de Mestrado poderão ser considerados para progressão ao curso de Doutorado.

	\item Serão habilitados a realizarem a matrícula nos cursos de Mestrado ou Doutorado aqueles candidatos inscritos para seleção segundo edital específico e selecionados pelo Colegiado segundo critérios publicados no mesmo edital.
	\item O edital de seleção deve ser proposto, aprovado e divulgado pelo Colegiado, observando requisitos legais quanto a prazos de divulgação e inscrição.
	\item Ressalvada as situações de existência de bolsas concedidas por agências de fomento ou de outras fontes diretamente aos orientadores, a alocação das bolsas aos candidatos matriculados, será feita pelo Colegiado, de forma competitiva entre os candidatos, por meio de um processo classificatório, e tomando como base as instruções e exigências das agências de financiamento do Programa.
	\begin{enumerate}
		\item Poderão participar do processo classificatório todos os alunos regularmente matriculados no Programa.
	\end{enumerate}
	
	\item No ato da matrícula, o candidato deverá apresentar toda a documentação estabelecida em resolução própria e especificada no edital de seleção.
	\begin{enumerate}
		\item No ato da primeira matrícula o candidato deverá, juntamente com seu Orientador designado, apresentar ao Colegiado seu Plano de Estudo.
		\item O Plano de Estudos deverá ser aprovado pelo Colegiado.
		\item Eventuais alterações no Plano de Estudos deverão ser discutidas e aprovadas pela Comissão de Acompanhamento Discente.
	\end{enumerate}

	\item A renovação de matrícula será feita a cada período letivo regular, até a Defesa da Dissertação ou Tese, sendo considerado desistente do curso o aluno que não a fizer.
	\begin{enumerate}
		\item A cada renovação da matrícula, o aluno, em acordo com seu Orientador, pode solicitar para apreciação pelo Colegiado a alteração das disciplinas previstas no Plano de Estudo original.
		\item Junto ao pedido de renovação de matrícula deve ser encaminhado Relatório de Andamento das atividades do aluno junto a parecer do Orientador e nota de ciência dos membros da Comissão de Acompanhamento Discente.
	\end{enumerate}

	\item O aluno que, por motivo de força maior, necessitar interromper seus estudos, poderá solicitar ao Coordenador do Programa, por escrito, o trancamento de sua matrícula, devendo o pedido ser acompanhado do parecer do Orientador.
	\begin{enumerate}
		\item Se for o caso, o pedido de trancamento deverá ser renovado a cada semestre.
		\item O aluno poderá trancar sua matrícula por um período máximo de um ano.
	\end{enumerate}

	\item Com a matrícula, o aluno assume o compromisso de submeter-se ao presente Regimento e aos demais Regimentos e Estatutos desta Instituição.
	\begin{enumerate}
		\singleitem O aluno deve zelar pelo patrimônio do Programa e da Universidade e pelo uso dos recursos que lhe forem oferecidos apenas para fins acadêmicos.
	\end{enumerate}
\end{enumerate}

\section{Do Regime Didático}
\begin{enumerate}
	\item O ensino é ministrado por meio de disciplinas, a cargo dos Docentes do Programa de Pós-Graduação em Computação.
	\item A unidade de integralização curricular será o Crédito, que corresponde a 17 horas aula, ou outras atividades definidas neste Regimento.
	\item O Colegiado do Curso poderá aceitar o aproveitamento de créditos obtidos em disciplinas de outros cursos de Pós-Graduação Stricto Sensu, desde que estejam relacionados à área de formação do aluno no Programa.
	\begin{enumerate}
		\singleitem O pedido de aproveitamento deverá ser encaminhado pelo aluno, com o parecer do Orientador. A equivalência das disciplinas cursadas em outros programas será julgada pelo Colegiado.
	\end{enumerate}
	\item Em cada disciplina, os mestrandos serão avaliados pelo Professor Responsável aplicando critérios previamente definidos, que poderão incluir um ou mais dos seguintes instrumentos: provas escritas, trabalhos escritos individuais ou em grupo, avaliações orais e participação em aulas (a qual inclui assiduidade, empenho e qualidade das contribuições do aluno). Com base nestes critérios, o Professor Responsável atribuirá a cada aluno um conceito variando de A a D.
	\item O aproveitamento do aluno em cada disciplina será expresso pelos seguintes conceitos, correspondendo às respectivas classes:
	\begin{itemize}
		\item A: 9,0 a 10,0
		\item B: 7,5 a 8,9
		\item C: 6,0 a 7,4
		\item D: abaixo de 5,9
		\item I: incompleto - atribuído ao aluno que, por motivo de força maior, for impedido de completar as atividades da disciplina no período regular;
		\item S: satisfatório - atribuído no caso das disciplinas Seminários, Exame de Qualificação, Estágio Docência, disciplinas de nivelamento e outras definidas pela Câmara de Pós-Graduação ``Stricto Sensu'';
		\item N: não-satisfatório - atribuído no caso das disciplinas Seminários, Exame de Qualificação, Estágio Docência, disciplinas de nivelamento e outras definidas pela Câmara de Pós-Graduação ``Stricto Sensu'';
		\item J: cancelamento - atribuído ao aluno que, com autorização do seu orientador, cancelar a matrícula na disciplina;
		\item T: trancamento - atribuído ao aluno que, com autorização do seu orientador e/ou com aprovação do Colegiado do Programa, tiver realizado o trancamento de matrícula;
		\item P: aproveitamento de créditos - atribuído ao aluno que tenha obtido aproveitamento de créditos realizados em outro Programa.
	\end{itemize}
	\begin{enumerate}
		\item Será considerado aprovado na Disciplina e terá direito a Crédito o aluno que obtiver um conceito A, B ou C.
		\item Será reprovado sem direito a Crédito o aluno que obtiver o conceito D.
	\end{enumerate}

	\item A avaliação do aproveitamento, ao término de cada período letivo, será feita por meio de média ponderada (coeficiente de rendimento), tomando-se como peso o número de créditos das disciplinas e atribuindo-se aos conceitos A, B, C, D os valores 4,0; 3,0; 2,0; e 0,0, respectivamente.

	\begin{enumerate}
		\item O conceito D será computado para cálculo do coeficiente de rendimento enquanto outro conceito não for atribuído à disciplina repetida.
		\item As disciplinas com conceito I, S, N, J, T ou P não serão consideradas no cômputo do coeficiente de rendimento.
	\end{enumerate}

	\item Estará automaticamente desligado do Programa o aluno que se enquadrar em uma ou mais das seguintes situações:
	\begin{enumerate}[label=\Roman*]
		\item Obtiver coeficiente de rendimento inferior a 2,0 no seu primeiro período letivo;
		\item Obtiver coeficiente de rendimento acumulado inferior a 2,5 no seu segundo período letivo e subseqüentes;
		\item Obtiver coeficiente de rendimento acumulado inferior a 3,0 no seu terceiro período letivo e subseqüentes;
		\item Obtiver conceito D em disciplina repetida;
		\item Não completar todos os requisitos do curso no prazo estabelecido;
		\item Não solicitar renovação do trancamento de matrícula, quando for o caso;
		\item Não atender outras exigências estabelecidas pelo Programa em seu Regimento.
	\end{enumerate}

	\item  Obrigatória a frequência a pelo menos 75\% das atividades da Disciplina.
	\begin{enumerate}
		\singleitem Receberá conceito D na Disciplina o aluno que faltar a mais de 25\% das aulas.
	\end{enumerate}

	\item O aluno para concluir seu curso deve ter aprovação nas Disciplinas Obrigatórias.
	\begin{enumerate}
		\item Em caso de reprovação em uma Disciplina Obrigatória, o aluno deverá cursá-la novamente quando de sua reedição, sendo desligado do Programa em uma segunda reprovação.
		\item A dispensa da realização de uma ou mais Disciplinas Obrigatórias pode ser solicitada pelo aluno caso este seja aprovado em Prova de Proficiência específica aplicada pelo Programa. No caso de dispensa concedida, os créditos relativos à disciplina não serão contabilizados para conclusão do curso.
	\end{enumerate}
\end{enumerate}

\section{Do Mestrado}
\begin{enumerate}
	\item A permanência mínima e máxima dos mestrandos no Programa de Mestrado será, respectivamente, de 12 meses e 30 meses, contados a partir da data da primeira matrícula.
	\begin{enumerate}
		\singleitem O prazo máximo estabelecido neste Artigo poderão ser prorrogados excepcionalmente por até seis meses, por recomendação do Orientador, com aprovação do Colegiado, caso o Mestrando tenha cumprido todos os requisitos, exceto a apresentação da Dissertação.
	\end{enumerate}

	\item Para solicitar a Defesa da Dissertação, o Mestrando deverá ter cumprido os seguintes pré-requisitos:
	\begin{enumerate}[label=\Roman*]
		\item 	Estar matriculado no Programa há pelo menos 12 meses;
		\item 	Ter completado pelo menos 20 créditos;
		\item 	Ter tido uma Proposta de Dissertação de Mestrado aprovada;
		\item 	Ter sido aprovado em um Seminário de Andamento de Dissertação de Mestrado;
		\item 	Entregar um exemplar da Dissertação de Mestrado ao Colegiado;
		\item 	Ter autorização do Orientador e ciência da Comissão de Acompanhamento Discente para marcar a Defesa.
	\end{enumerate}

	\item Será exigido dos mestrandos proficiência em Língua Inglesa, a qual deverá obrigatoriamente ser apresentada até a quarta matrícula no Programa.
	\begin{enumerate}
		\singleitem O Exame de Proficiência (competência) deverá ser realizado por entidade reconhecida pelo Colegiado do Programa.
	\end{enumerate}

	\item A redação e formatação da Dissertação deverão observar as normas estabelecidas pela Universidade Federal de Pelotas.

	\item A Proposta de Dissertação será avaliada por relator indicado pela Comissão de Acompanhamento Discente.
	\begin{enumerate}
		\item A defesa da Dissertação não poderá ocorrer antes de transcorridos ao menos seis meses da aprovação da Proposta de Dissertação.
		\item Caso a Proposta de Dissertação seja reprovada, o Mestrando deve apresentar nova Proposta no prazo especificado pelo relator.
	\end{enumerate}

	\item O Seminário de Andamento será avaliado por banca, indicada pela Comissão de Acompanhamento Discente, em uma sessão pública.
	\begin{enumerate}
		\item A defesa da Dissertação não poderá ocorrer antes de transcorridos ao menos quatro meses da aprovação no Seminário de Andamento.
		\item Caso reprovado no Seminário de Andamento, novo o Mestrando deve apresentar novo Seminário no prazo especificado pela banca.
	\end{enumerate}

	\item O mestrando que, tendo sido aprovado pela banca examinadora na defesa de Dissertação de Mestrado e cumprido os demais requisitos especificados neste Regimento, estará credenciado a receber o grau de Mestre em Ciência da Computação.

\end{enumerate}

\section{Alteração de nível mestrado para doutorado}
\begin{enumerate}
	\item  A alteração do nível de mestrado para o de doutorado será permitida a alunos que contemplem os seguintes requisitos:
	\begin{enumerate}[label=\Roman*]
		\item Ter cursado no mínimo dois semestres no Programa;

		\item Apresentar Coeficiente de Rendimento igual ou superior a 3,5;

		\item Apresentar solicitação de alteração na inscrição em formulário próprio dentro do calendário do programa, preenchido pelo orientador, devidamente justificada;

		\item Apresentar relatório de atividades do período em que está no mestrado e projeto para o doutorado.
	\end{enumerate}

	\begin{enumerate}
		\singleitem O Colegiado indicará uma comissão que avaliará o mérito da solicitação. Em caso de aprovação, o estudante terá um prazo de 90 dias para defender a dissertação.
	\end{enumerate}

	\item Em casos especiais, a critério do Colegiado, durante a realização do Mestrado em Ciência da Computação será permitida a alteração da inscrição para Doutorado, com o aproveitamento dos créditos já obtidos.	

\end{enumerate}

\section{Do Doutorado}
\begin{enumerate}
	\item  A permanência mínima e máxima dos doutorandos no curso de Doutorado será, respectivamente, de 24 meses e 54 meses, contados a partir da data da primeira matrícula.
	\begin{enumerate}
		\singleitem Os prazos máximos estabelecidos neste Artigo poderão ser prorrogados excepcionalmente por até seis meses, por recomendação do Orientador, com aprovação do Colegiado, caso o Doutorando tenha cumprido todos os requisitos, exceto a apresentação da Tese.
	\end{enumerate}
	
	\item Para solicitar a Defesa da Tese, o doutorando deverá ter cumprido os seguintes pré-requisitos:
	\begin{enumerate}[label=\Roman*]
		\item	Estar matriculado no Programa há pelo menos 24 meses;
		\item	Ter completado pelo menos 36 créditos, dois dos quais em Trabalho Individual;
		\item	Ter tido uma Proposta de Tese de Doutorado aprovada;
		\item	Ter sido aprovado em um Seminário de Andamento de Tese de Doutorado;
		\item	Ter sido aprovado em Exame de Qualificação;
		\item	Entregar um exemplar da Tese de Doutorado ao Colegiado;
		\item	Ter autorização do Orientador e ciência da Comissão de Acompanhamento Discente para marcar a Defesa.
		\item	Ter produção científica no tema da Tese de Doutorado, desenvolvida durante o Doutorado, conforme estabelecido em Resolução específica pelo Colegiado.
	\end{enumerate}
	\begin{enumerate}
		\singleitem Créditos de Trabalhos Individuais não podem ser reaproveitados ou revalidados.
	\end{enumerate}

	\item  Será exigida dos doutorandos proficiência, comprovada por entidade reconhecida pelo Programa, em Língua Inglesa.

	\item  O aluno deverá ser aprovado em um Exame de Qualificação, que avaliará os conhecimentos do aluno nas áreas necessárias à Tese, em prazo e formato definidos pelo Colegiado através de Resolução própria;
	\begin{enumerate}
		\item O Exame de Qualificação será avaliado por uma Banca de Avaliação de Exame, com membros definidos pelo Colegiado;
		\item No caso de reprovação no Exame de Qualificação, o aluno poderá prestar um único novo exame, em período máximo estipulado pela Banca de Avaliação de Exame;
		\item A reprovação em dois Exames de Qualificação ou a não prestação do Exame no prazo estabelecido levará ao desligamento do aluno do Programa.
	\end{enumerate}

	\item  A redação e formatação da Tese deverão observar as normas estabelecidas pela Universidade Federal de Pelotas.
	\item  A Proposta de Tese de Doutorado será avaliada por relator indicado pela Comissão de Acompanhamento Discente.
	\begin{enumerate}
		\item Caso a Proposta de Tese seja reprovada, o doutorando deve apresentar nova Proposta nos prazos especificados pelos avaliadores.
	\end{enumerate}
	\item  O doutorando que, tendo sido aprovado pela banca examinadora na defesa de Tese de Doutorado e cumprido os demais requisitos especificados neste Regimento, estará credenciado a receber o grau de Doutor em Ciência da Computação.
\end{enumerate}

\section{Da Defesa de Dissertação ou Tese}

\begin{enumerate}
	\item Defesas de Dissertação ou Tese serão de caráter público, perante Banca Examinadora, constituída de no mínimo três membros, presidida pelo Orientador. Os outros membros serão professores com título de doutor, sendo composta por, pelo menos um (1) membro externo ao Programa e um (1) membro pertencente ao Programa.
	\begin{enumerate}
		\item Ao final da Defesa, a Banca Examinadora preencherá uma Ata de Defesa, onde constará o parecer final sobre o conceito atribuído à Dissertação ou Tese apresentada e as solicitações de correções necessárias para homologação final do documento.
		\item Em casos excepcionais, quando há interesse em proteger o conhecimento gerado em função de pedido de patente, a Defesa poderá ser de caráter sigiloso, desde que solicitado pelo Orientador e seu Orientando e recebida aprovação do Colegiado.
		\item É vedado ao Presidente da Banca Examinadora emitir parecer sobre o trabalho apresentado.
	\end{enumerate}

	\item Compete ao Colegiado do Programa homologar a decisão da Banca Examinadora, após parecer do Orientador sobre o atendimento das correções solicitadas na Ata de Defesa.
	\begin{enumerate}
		\singleitem A Ata de Defesa deverá conter as alterações obrigatórias a serem feitas na Dissertação, bem como o prazo para a realização das mesmas, e as assinaturas de todos os membros da Banca Examinadora.
	\end{enumerate}
	\item Após a Defesa, e dentro dos prazos especificados na Ata de Defesa, o aluno deverá encaminhar à Secretaria do Programa, para homologação, um exemplar impresso da Dissertação ou Tese corrigida e três cópias em CD. Estas cópias são destinadas à biblioteca do Programa, para Pró-Reitoria de Pesquisa e Pós-Graduação e para o próprio Programa. O material entregue deverá ser acompanhado de aprovação por escrito do Orientador ou do membro indicado da Banca Examinadora na própria Ata de Defesa, conforme o caso.

\end{enumerate}

\section{Da Representação Discente}

\begin{enumerate}
	\item A representação discente junto ao Colegiado do Programa será exercida por alunos matriculados de forma regular, eleitos por seus pares, com mandato de um (1) ano.
	\begin{enumerate}
		\item O número de representantes discentes será estabelecido de acordo com o Regimento Geral da Universidade.
		\item Deverá haver representantes dos cursos de Mestrado e Doutorado, em igual número.
		\item O voto dos representantes discentes junto ao Colegiado é universal.
		\item Haverá um suplente para o cargo de representante discente.
		\item O Suplente assumirá suas funções em caso de licença, afastamento temporário ou renúncia do membro titular.
	\end{enumerate}
	\item O Representante Discente é responsável por realizar o processo de votação para indicação seu sucessor e seu respectivo suplente.
	\begin{enumerate}
		\singleitem Na ausência, por qualquer motivo, de representação discente, cabe ao Colegiado realizar o processo de votação para escolha do Representante Discente e seu suplente.
	\end{enumerate}
\end{enumerate}

\section{Das Disposições Geras e Transitórias}

\begin{enumerate}
	\item As decisões ad referendum deverão ser submetidas à homologação do Colegiado em reunião subseqüente, obedecidos seus prazos normais de ocorrência.
	\item Os casos omissos neste Regimento serão resolvidos pelo Colegiado, respeitando o Regimento Geral dos Cursos de Pós-Graduação Stricto Sensu.
	\begin{enumerate}
		\singleitem O Regimento Geral de Cursos de Pós-Graduação Stricto Sensu e ao Regimento Geral da Pró-Reitoria de Pesquisa e Pós-Graduação devem ser consultados para casos omissos ao presente Regimento. 
	\end{enumerate}
	\item  O presente regimento passará a vigorar a partir de sua aprovação pelo Conselho Coordenador de Ensino, Pesquisa e Extensão desta Instituição.

\end{enumerate}






\end{document}