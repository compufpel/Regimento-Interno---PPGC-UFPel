
% Capitulos sao secoes
% Artigos e paragrafos sao enumerates

\documentclass{article}
\usepackage{enumitem}
\usepackage[utf8]{inputenc}
\usepackage{titlesec}
\usepackage{xspace} 
\usepackage[brazil]{babel}
\usepackage[T1]{fontenc}

\titleformat{\section}{\normalfont\scshape}{Capítulo \Roman{section}}{1em}{}

\setlist[enumerate,1]{label=Art. \theenumi,resume}

\setlist[enumerate,2]{label=§ \arabic*$^{\circ}$,align=left}

\newcommand{\singleitem}{\item[Parágrafo Único.]}

\newcommand{\grupoMenor}{Colegiado\xspace}
\newcommand{\grupoMaior}{Conselho\xspace}
%\hyphenize{a\-com\-pa\-nha\-men\-to}

\title{\large{Universidade Federal de Pelotas}\\Centro de Desenvolvimento Tecnológico\\Programa de Pós-Graduação em Computação\\\vspace{5mm}\Large{\textbf{Regimento Interno}}}

\date{}



\begin{document}

\maketitle

\vspace{-10mm}

\tableofcontents


\section{Das Finalidades}

\begin{enumerate}
\item O Programa de Pós-Graduação em Computação da Universidade Federal de Pelotas, neste documento referenciado por PPGC ou simplesmente por Programa, tem por finalidade a formação de recursos humanos capazes de promover o avanço científico e tecnológico da área de Computação pela atuação no ensino e realização de pesquisas científicas.

\item O PPGC oferece cursos no nível de Mestrado Acadêmico e Doutorado, ambos na área de Ciência da Computação.
\begin{enumerate}
	\item O curso de Mestrado Acadêmico em Ciência da Computação no PPGC tem como objetivo prover formação em uma linha específica de estudos, habilitando os alunos a planejar, executar, reportar e aplicar projetos de pesquisa relevantes à área de Ciência da Computação.

	\item O curso de Doutorado em Ciência da Computação no PPGC tem como objetivo desenvolver competências técnicas e científicas em profundidade em uma linha específica de estudos, que permitam ao aluno contribuir de forma significativa para o avanço acadêmico e científico da área de Ciência da Computação pelo planejamento, execução, disseminação e aplicação de projetos de pesquisa originais.

\end{enumerate}

\end{enumerate}

\section{Da Administração do Programa}

\begin{enumerate}

	\item A administração do Programa é exercida por:
	\begin{enumerate}[label=\Roman*]	
		\item \grupoMaior do Programa, neste documento referenciado apenas por \grupoMaior, com funções deliberativas e consultivas;
		\item \grupoMenor do Programa, neste documento referenciado apenas por \grupoMenor, com funções deliberativas, normativas e executivas;
		\item Coordenador e Coordenador Adjunto, com funções executivas.
	\end{enumerate}

	\item O \grupoMaior é composto pelos Representantes Discentes e todos docentes Permanentes e Colaboradores do Programa pertencentes ao quadro funcional da Universidade Federal de Pelotas.

	\item O \grupoMenor é composto pelos seguintes membros:
	\begin{enumerate}[label=\Roman*]
		\item Coordenador do Programa;
		\item Coordenador Adjunto do Programa;
		\item Docentes do Programa em número igual a quarta parte do total de docentes, arredondado para o número inteiro superior;
		\item Representante Discentes.
	\end{enumerate}

	\item Os membros docentes do \grupoMenor são eleitos pelo \grupoMaior por meio de votação.
	\begin{enumerate}
		\item Cada Linha de Pesquisa do Programa terá direito a uma posição no \grupoMenor, na forma do seu membro com maior número de votos.
		\item As demais posições serão preenchidas na ordem especificada na votação, do docente com maior número de votos para o com menor número.
		\item A suplência será designada conforme aplicação das regras deste Artigo dentre os candidatos sem mandato atribuído.
		\item Os membros docentes do \grupoMenor terão mandato de três anos.		
	\end{enumerate}

	\item O \grupoMaior e o \grupoMenor são presididos pelo Coordenador do Programa ou, na ausência deste, pelo Coordenador Adjunto, seguido pelo membro mais antigo na Instituição pertencente ao \grupoMenor.
	\begin{enumerate}
		\item O Coordenador e o Coordenador Adjunto são eleitos pelo \grupoMaior dentre seus membros, conforme regimento institucional.
		\item Os mandatos serão de 2 anos, sendo permitida apenas uma recondução sucessiva aos cargos.
	\end{enumerate}

	\item Os Representantes Discentes, em número de 1 para cada curso, são eleitos pelos alunos regulares dos respectivos cursos, para mandatos de 1 ano.
	\begin{enumerate}
		\item Os votos dos Representantes Discentes são universais.
		\item O aluno mais votado assumirá a função de Representante Discente enquanto o segundo aluno mais votado assumirá a função de suplente do Representante Discente.
		\item O processo de eleição é dever do \grupoMenor.
	\end{enumerate}	

	\item Assessoram nas decisões do \grupoMaior e do \grupoMenor, três Comissões Permanentes, cujos membros são indicados pelo \grupoMaior entre os membros do Corpo Docente do Programa:
	\begin{enumerate}[label=\Roman*]
		\item Comissão de Acompanhamento Discente;
		\item Comissão de Avaliação do Programa;
		\item Comissão de Seleção e Bolsas.
	\end{enumerate}

\end{enumerate}

\section{Das Atribuições do \grupoMaior}
\begin{enumerate}

	\item O \grupoMaior reunir-se-á quando convocado pelo seu Coordenador ou por, no mínimo, metade dos seus membros.
	\begin{enumerate}
		\singleitem Decisões do \grupoMaior só terão validade quando presente, no momento da decisão, no mínimo metade dos seus membros.
	\end{enumerate}

	\item O \grupoMaior deliberará por maioria simples de votos dos membros presentes.
	\begin{enumerate}
		\item O Coordenador não possui direito a voto.
		\item Todos os demais membros possuem direito a voto, em igual peso.
		\item O Coordenador dará o Voto de Qualidade em caso de empate em votações. 
	\end{enumerate}

	\item Compete ao \grupoMaior:
	\begin{enumerate}[label=\Roman*]
		\item Eleger o Coordenador e o Coordenador Adjunto;
		\item Eleger os membros do \grupoMenor;
		\item Promover alterações no Regimento do Programa, para aprovação pela Câmara de Pós-Graduação ``Stricto Sensu'' e pelos demais órgãos competentes;
		\item Estabelecer metas de curto, médio e longo prazos para o Programa;
		\item Julgar, em grau de recurso, decisões proferidas pelo Coordenador do Programa e pelo \grupoMenor;
		\item Avaliar e apresentar julgamento sobre pedidos de credenciamento de docentes junto ao Programa;
		\item Analisar e se pronunciar sobre relatórios, pedidos e sugestões advindos das Comissões Permanentes;
		\item Se pronunciar sobre prioridades de aplicação de recursos específicos do Programa;
		\item Definir a Meta Individual Mínima e a Meta Global Mínima para fins de processos de avaliação dos docentes;
		\item Estabelecer a regularidade dos processos seletivos para novos discentes;
		\item Deliberar sobre assuntos de interesse do Programa;		
		\item Propor aos órgãos competentes da Universidade a interrupção, suspensão ou cessação das atividades do Programa ou de seus cursos, quando cabível.
	\end{enumerate}

	\begin{enumerate}
		\singleitem Recursos às decisões do \grupoMaior devem ser dirigidos à Câmara de Pós-Graduação ``Stricto Sensu'' da Pró-reitoria de Pesquisa e Pós-Graduação desta Universidade.
	\end{enumerate}
\end{enumerate}

\section{Das Atribuições do \grupoMenor}
\begin{enumerate}

	\item O \grupoMenor reunir-se-á mensalmente ou quando convocado por qualquer dos seus membros.
	\begin{enumerate}
		\singleitem Decisões do \grupoMenor só terão validade quando presente, no momento da decisão, no mínimo metade dos seus membros.
	\end{enumerate}

	\item O \grupoMenor do Programa deliberará por maioria simples de votos dos membros presentes.
	\begin{enumerate}
		\item O Coordenador não possui direito a voto.
		\item Todos os demais membros possuem direito a voto, em igual peso.
		\item O Coordenador dará o Voto de Qualidade em caso de empate na votação. 
	\end{enumerate}

	\item Compete ao \grupoMenor do Programa:
	\begin{enumerate}[label=\Roman*]
		%\item Executar as diretrizes estabelecidas pela Pró-Reitoria de Pesquisa e Pós-Graduação e pelo Conselho Coordenador de Ensino, Pesquisa e Extensão desta Instituição; % Que diretrizes?
		\item Garantir a execução e cumprimento do Regimento do Programa;
		\item Estabelecer normas necessárias ao bom andamento do Programa;
		\item Propor ao \grupoMaior mudanças no Regimento do Programa;
		\item Estabelecer estratégias e ações para atingir as metas estabelecidas pelo \grupoMaior;
		\item Conciliar os interesses do Programa com os da Unidade na qual se insere;
		\item Garantir a atualização das informações didáticas do Programa;
		\item Fixar a sequência recomendável de estudos e os pré-requisitos necessários;
		\item Emitir parecer sobre assuntos de interesse do Programa de Pós-Graduação;
		\item Analisar e emitir parecer sobre os pedidos de transferência, aproveitamento de estudos e adaptações, de acordo com as normas vigentes;
		\item Julgar, em grau de recurso, decisões proferidas pelo Coordenador do Programa;
		\item Verificar o cumprimento do Conteúdo Programático e da Carga Horária das disciplinas dos cursos;
		\item Indicar, à ocasião do Processo Seletivo de novos ingressantes, os Orientadores para cada candidato selecionado;	
		\item Acolher, avaliar, solicitar alterações e aprovar o Plano de Estudo de cada aluno;
		\item Realizar o acompanhamento dos alunos por meio de registros individuais;
		\item Homologar a nominata para Banca Examinadora de cada pedido de Defesa de Dissertação ou Defesa de Tese recebido;
		\item Homologar as dissertações e teses após a banca de defesa e após ter sido comprovada a realização das correções exigidas pela banca examinadora, se alguma;
		\item Realizar regularmente a avaliação do Corpo Docente promovendo o descredenciamento de membros junto a Programa segundo as normas estabelecidas;
		\item Analisar e se pronunciar sobre relatórios, pedidos e sugestões advindos das Comissões Permanentes;
		\item Indicar comissões temporárias de assessoramento, conforme demanda;
		\item Dar o encaminhamento devido, no tempos regulamentares, aos procedimentos solicitados pelas Comissões estabelecidas;
		\item Resolver, nos limites de sua competência, os casos omissos deste Regimento.
	\end{enumerate}

	\begin{enumerate}
		\singleitem Recursos às decisões do \grupoMenor devem ser dirigidos ao \grupoMaior.
	\end{enumerate}
\end{enumerate}

\section{Das Atribuições do Coordenador do Programa}
\begin{enumerate}
	\item Ao Coordenador de Programa, compete:
	\begin{enumerate}[label=\Roman*]
		\item Coordenar e supervisionar o funcionamento do Programa;
		\item Convocar e presidir as reuniões do \grupoMaior e do \grupoMenor;
		\item Representar o Programa e as decisões tomadas no \grupoMaior e no \grupoMenor;
		\item Implementar a oferta das disciplinas necessárias ao andamento dos cursos;
		\item Estabelecer a distribuição de carga-horária entre os docentes;
		\item Estabelecer a demanda e a distribuição de bolsas entre os discentes e informar os órgãos competentes;
		\item Elaborar, quando requisitado, relatórios destinados às instituições fornecedoras de bolsas;
		\item Comunicar ao órgão competente qualquer irregularidade no funcionamento do Programa e solicitar as correções necessárias;
		\item Designar Relator ou Comissão para estudo de matéria submetida ao \grupoMaior ou ao \grupoMenor;
		\item Decidir ``ad referendum'' sobre matéria de urgência do \grupoMenor ou \grupoMaior;
		\item Exercer outras atribuições inerentes ao cargo;
		\item Supervisionar e zelar pela aplicação das verbas específicas do Programa.
	\end{enumerate}
	\item Ao Coordenador Adjunto de Programa, compete Substituir o Coordenador em suas ausências ou impedimentos, auxiliá-lo na execução das deliberações do \grupoMaior e do \grupoMenor e executar as tarefas que lhe forem especificamente designadas pelo \grupoMenor.
\end{enumerate}

\section{Das Atribuições das Comissões Permanentes}
\begin{enumerate}
	\item Compete à Comissão de Acompanhamento Discente:
	\begin{enumerate}[label=\Roman*]
		\item Monitorar, ao final de cada período letivo, o desempenho acadêmico dos discentes do Programa;
		\item Identificar discentes em risco de não cumprir os requisitos necessários à conclusão do curso e levar estes casos à Coordenação e aos Orientadores relevantes aos casos;
		\item Propor aos Orientadores, quando relevante, ações corretivas a discentes em risco;
		\item Identificar discentes que estejam fora das normas estabelecidas para o regime acadêmico ou o regime de manutenção de bolsas, levando estes casos à Coordenação.
	\end{enumerate}

	\item Compete à Comissão de Avaliação:
	\begin{enumerate}[label=\Roman*]
		\item Estabelecer métricas relevantes ao acompanhamento da qualidade do Programa, considerando regulamentos dos órgãos que regem a pós-graduação no país;
		\item Estabelecer valores desejáveis às métricas estabelecidas, considerando as metas de curto, médio e longo prazos definidas pelo \grupoMaior;
		\item Monitorar as métricas estabelecidas, com frequência mínima anual;
		\item Relatar ao \grupoMenor os resultados do monitoramento;
		\item Propor ações para melhoria da qualidade do Programa.
	\end{enumerate}

	\item Compete à Comissão de Seleção e Bolsas:
	\begin{enumerate}[label=\Roman*]
		\item Propor ao \grupoMenor critérios específicos de seleção de novos discente ao Programa;
		\item Elaborar o Edital de Seleção para cada processo seletivo de novos discentes ao Programa;
		\item Homologar as inscrições dos processos seletivos;
		\item Conduzir o processo seletivo e submeter os resultados ao \grupoMenor;
		\item Elaborar e manter documentação dos processos seletivos;
		\item Estabelecer os critérios de elegibilidade e a alocação das bolsas disponíveis, bem como os critérios de manutenção das bolsas pelos beneficiados.
	\end{enumerate}
\end{enumerate}

\section{Do Corpo Docente}

\begin{enumerate}
	\item O Corpo Docente do PPGC é constituído por professores e pesquisadores, denominados Docentes, portadores de título de doutor devendo ser composto majoritariamente por integrantes do quadro funcional da Universidade Federal de Pelotas.

	\begin{enumerate}
		\item Poderão integrar o Corpo Docente do Programa, pesquisadores de outras Instituições de Ensino Superior, nacionais ou estrangeiras, de centros de pesquisa do país ou do exterior.

		\item Solicitações de credenciamento junto ao Corpo Docente do Programa podem ser encaminhadas a qualquer momento ao \grupoMaior.
	\end{enumerate}

	% \item Para efeito de enquadramento junto ao Programa, define-se como Docente Ativo o docente que, em um dado ano, tenha atuado no Programa realizando duas ou mais atividades naquele ano, entre: (a) lecionar uma disciplina, (b) orientar ou co-orientar uma dissertação de mestrado ou tese de doutorado, (c) participar na produção científica qualificada do Programa. 

	\item Os Docentes serão enquadrados como:

	\begin{enumerate}[label=\Roman*]

		\item Permanentes: pesquisadores vinculados à Universidade Federal de Pelotas e, de forma integral ou parcial, à Unidade Acadêmica a qual pertence o Programa.

		\item Visitantes: identificados por possuirem vínculo com alguma instituição, no Brasil ou no Exterior, que permanecerem, durante um período contínuo e determinado, à disposição do Programa, contribuindo para o desenvolvimento das atividades acadêmico-científicas deste.	
		
		\item Colaboradores: demais pesquisadores credenciados junto ao Programa.

	\end{enumerate}

	\item São obrigações dos Docentes do Programa:
	\begin{enumerate}[label=\Roman*]
		\item 	Desenvolver projetos de pesquisa que contribuam para a área de Computação;
		\item	Orientar ou Co-orientar Dissertações de Mestrado e Teses de Doutorado de alunos do Programa, quando designado pelo \grupoMenor;
		\item	Ministrar aulas teóricas e práticas de disciplinas do Programa, de acordo com o programa vigente de cada Disciplina, quando designado pelo Coordenador;
		\item	Manter o Registro Acadêmico da Disciplina, bem como o Registro de Desempenho individual de cada aluno nela inscrito;
		\item	Atualizar e divulgar o programa da disciplina a cada edição desta;
		\item	Promover e participar de seminários, simpósios e estudos dirigidos;
		\item	Participar de bancas e comissões, quando designado pelo \grupoMenor;
		\item	Responder à Comissão de Avaliação do Programa quando solicitado;
		\item	Divulgar resultados de suas pesquisas nos meios próprios;
		\item	Acatar as decisões do \grupoMaior e do \grupoMenor e executar as tarefas que nestes fóruns lhe forem atribuídas no prazo conveniado;
		\item	Desempenhar demais atividades, dentro dos dispositivos regimentais, que possam beneficiar o Programa.
	\end{enumerate}

	\begin{enumerate}
		\item O Docente que no ato do processo de avaliação anual não tiver atingido a Meta Individual Mínima definida pelo \grupoMaior, não poderá acolher novos Orientandos.

		\item O Docente que em dois anos consecutivos não atingir a Meta Individual Mínima não terá disciplina a ele atribuída.
	\end{enumerate}
\end{enumerate}

\section{Do Credenciamento e Descredenciamento Docente}
\label{credenciamento}
\begin{enumerate}
	\item Docentes serão considerados para credenciamento junto ao Programa mediante pedido por escrito ao \grupoMaior;

	\item O solicitante deverá atender aos requisitos mínimos definidos em resolução própria do Programa e deverá ser aprovado pelo \grupoMaior;

	\item Docentes credenciados serão desligados do Programa quando:
	\begin{enumerate}
		\item Durante três avaliações consecutivas, não atingir a Meta Individual Mínima definida pelo \grupoMaior, exceto nos casos onde:
		\begin{enumerate}
			\item A Meta Global Mínima do Programa no ano da avaliação for atingida; ou,
			\item O Docente possuir orientações em andamento.
		\end{enumerate}

		\item Durante seis avaliações consecutivas não atingir a Meta Individual Mínima.

		\item As avaliações serão realizadas no primeiro trimestre de cada ano.
	\end{enumerate}

\end{enumerate}


\section{Da Orientação}

\begin{enumerate}
	\item Cada Aluno ingressante no Programa contará com um Orientador e deverá se reportar à Comissão de Acompanhamento Discente.
	\begin{enumerate}
		\item Compete ao \grupoMenor determinar o Orientador a cada novo Aluno, observando as informações apresentadas em sua candidatura e a disponibilidade de orientação dos membros do Programa.
		\item A alteração de Orientação pode ser solicitada pelo Aluno ou pelo Orientador até a quarta matrícula no curso de Mestrado e a sétima matrícula no curso de Doutorado, sendo objeto de apreciação e parecer do \grupoMenor.
	\end{enumerate}

	\item Os Professores Orientadores do Programa são membros do Corpo Docente. 
	\begin{enumerate}
		\item O número máximo de orientações simultâneas em cada nível será determinada por regulamentação do \grupoMenor.
		\item Somente estão habilitados a orientar alunos de Doutorado os docentes que possuírem pelo menos uma orientação concluída, como orientador principal, de aluno de Mestrado ou Doutorado. neste ou em outro Programa.
	\end{enumerate}

	\item  São atribuições do Professor Orientador:
	\begin{enumerate}[label=\Roman*]
		\item 	Elaborar, juntamente com o aluno, seu Plano de Estudos e encaminhá-lo ao \grupoMenor dentro dos prazos regulamentares;
		\item	Orientar o aluno na execução do Plano de Estudos;
		\item	Promover o bom andamento do projeto de pesquisa do orientando;
		\item	Atuar na captação de recursos financeiros para custear o desenvolvimento dos projetos de pesquisa de seus orientandos;
		\item	Orientar, acompanhar e autorizar a matrícula dos seus orientandos a cada semestre;
		\item	Indicar ao \grupoMenor, se julgar conveniente, o(s) Coorientador(es) de seus orientandos;
		\item	Autorizar seus orientandos a defenderem suas Dissertações ou Teses;
		\item	Sugerir ao \grupoMenor os nomes dos integrantes de Bancas Examinadoras e a data para a realização das defesas de seus orientandos;
		\item	Presidir a Banca Examinadora de Defesa de seus orientandos.
	\end{enumerate}

	\item O pedido de inclusão de Co-orientador deve ser encaminhado pelo Orientador para apreciação e aprovação do \grupoMenor até a quarta matrícula no curso de Mestrado ou sétima matrícula no curso de Doutorado.
	\begin{enumerate}
		\item Na impossibilidade do Orientador presidir a Banca Examinadora, compete a um dos Coorientadores, presidi-la.
		\item É vedada a composição de uma Banca Examinadora com o Orientador e um ou mais dos eventuais Coorientadores, com papel de avaliador.
		%%%%%%%%%%%%%%%%%%%%%%%%% REVER %%%%%%%%%%%%%%%%%%%%%
	\end{enumerate}

\end{enumerate}

\section{Da Seleção e Matrícula}
\begin{enumerate}
	\item Serão habilitados a matricular-se nos cursos de Mestrado ou Doutorado aqueles candidatos inscritos para seleção segundo edital específico e selecionados pelo \grupoMenor segundo critérios publicados no mesmo edital.

	\begin{enumerate}

		\item Serão considerados para admissão ao curso de Mestrado candidatos que sejam portadores de diploma de graduação reconhecido por órgão competente.

		\item Serão considerados para admissão ao curso de Doutorado candidatos que sejam portadores de diploma de Mestre, modalidade Mestrado Acadêmico, reconhecido por órgão competente.

		\item Em caráter excepcional, a critério do \grupoMenor e por indicação de Docente do Programa, poderão ser considerados para admissão no curso de Doutorado candidatos sem título de Mestre.
	
		\item Em caráter excepcional, a critério do \grupoMenor e por requisição do Orientador, alunos matriculados no curso de Mestrado poderão ser considerados para progressão ao curso de Doutorado.
	\end{enumerate}

	\item Ressalvadas as situações de existência de bolsas concedidas por agências de fomento ou de outras fontes diretamente aos orientadores, a alocação das bolsas aos candidatos matriculados será feita pelo \grupoMenor, assessorado pela Comissão de Seleção e Bolsas, de forma competitiva entre os candidatos, por meio de processo classificatório, e tomando como base as instruções e exigências das agências de financiamento destas bolsas.
	\begin{enumerate}
		\item Poderão participar do processo classificatório todos os alunos regularmente matriculados no Programa.
	\end{enumerate}
	
	\item A renovação de matrícula será feita a cada período letivo regular, até a Defesa da Dissertação ou Tese, sendo considerado desistente do curso o aluno que não a fizer.
	\begin{enumerate}
		\singleitem Junto ao pedido de renovação de matrícula deve ser encaminhado Relatório de Andamento das atividades do aluno junto a parecer do Orientador e nota de ciência dos membros da Comissão de Acompanhamento Discente.
	\end{enumerate}

	\item O aluno que, por motivo de força maior, necessitar interromper seus estudos, poderá solicitar ao Coordenador do Programa, por escrito, o trancamento de sua matrícula, devendo o pedido ser acompanhado de ciência do Orientador.
	\begin{enumerate}
		\item Se for o caso, o pedido de trancamento deverá ser renovado a cada semestre, sob pena de perda do vínculo com o Programa.
		\item O aluno poderá trancar sua matrícula por um período máximo de um ano, em semestres consecutivos ou não.
	\end{enumerate}

	\item Com a matrícula, o aluno assume o compromisso de submeter-se ao presente Regimento e aos demais Regimentos e Estatutos desta Instituição, bem como zelar pelo patrimônio do Programa e da Universidade e pelo uso dos recursos que lhe forem oferecidos apenas para fins acadêmicos.
\end{enumerate}


\section{Do Regime Didático}
\begin{enumerate}
	\item O Regime Didático é composto por Atividades Obrigatórias, definidas na Seção \ref{atividades}, Disciplinas de Formação Básica, Disciplinas Eletivas, Disciplinas Obrigatórias e Atividades Complementares, definidas pelo \grupoMenor, especificadas em Plano de Estudos redigido pelo Aluno e seu Orientador.
	\begin{enumerate}
		\singleitem O Plano de Estudos deve ser aprovado pelo \grupoMenor e submetido a este ao longo do primeiro trimestre de ingresso do aluno no Curso.
	\end{enumerate}

	\item A unidade de integralização curricular é o Crédito, concedido em caso de aprovação.

	\item \label{disciplinas} Podem compor o Plano de Estudos, disciplinas ofertadas por Programas de Pós-Graduação Stricto Sensu de Instituições de Ensino Superior no país ou exterior.
	\begin{enumerate}
		\item Para Programas de Pós-Graduação no país, estes devem ser recomendados pela CAPES.
		\item \label{definicao-credito-aula} Um crédito é devido a cada 17 horas-aula.
	\end{enumerate}

	\item Disciplinas cursadas em momento anterior ao ingresso do aluno no Programa poderão ser aproveitadas para integralização do Plano de Estudos.
	\begin{enumerate}
		\item O aproveitamento de disciplinas cursadas em outros programas é limitado a 40\% do total de créditos do Plano de Estudos.
		\item É vedado o uso de recursos do Programa para subsidiar o acompanhamento de disciplinas em outras instituições.
		\item Somente poderão ser aproveitadas disciplinas cursadas há até 5 (cinco) anos da data de ingresso do aluno no Curso e cujos conceitos sejam A, B ou equivalente.
		\item A critério do \grupoMenor, uma ou mais disciplinas de outros programas poderão ser consideradas equivalentes a uma ou mais disciplinas do próprio Programa sendo, neste caso, devidos os créditos destas últimas.
		\item Disciplinas sem equivalência serão registradas com a sua denominação e carga horária originais e número de créditos convertido pela relação estabelecida no \ref{disciplinas}, \ref{definicao-credito-aula}.
	\end{enumerate}

	\item Em disciplinas ofertadas pelo Programa, os alunos serão avaliados pelo Professor Responsável aplicando critérios previamente definidos, dentre os quais devem estar incluídos um ou mais dos seguintes instrumentos: provas escritas, trabalhos escritos individuais ou em grupo, avaliações orais e participação em aulas (a qual inclui assiduidade, empenho e qualidade das contribuições do aluno). Com base nestes critérios, o Professor Responsável atribuirá a cada aluno um conceito variando de A a D.
	
	\item O aproveitamento do aluno em cada Disciplina será expresso pelos seguintes conceitos, correspondendo às respectivas classes:
	\begin{itemize}
		\item A: 9,0 a 10,0
		\item B: 7,5 a 8,9
		\item C: 6,0 a 7,4 
		\item D: abaixo de 5,9
		\item I: incompleto, atribuído ao aluno que, por motivo de força maior, for impedido de completar as atividades da disciplina no período regular;
		\item S: satisfatório, atribuído no caso das disciplinas Seminários, Exame de Qualificação, Estágio Docência, disciplinas de nivelamento e outras definidas pela Câmara de Pós-Graduação ``Stricto Sensu'';
		\item N: não-satisfatório, atribuído no caso das disciplinas Seminários, Exame de Qualificação, Estágio Docência, disciplinas de nivelamento e outras definidas pelo \grupoMenor ou pela Câmara de Pós-Graduação ``Stricto Sensu'';
		\item J: cancelamento, atribuído ao aluno que, com autorização do seu orientador, cancelar a matrícula na disciplina;
		\item T: trancamento, atribuído ao aluno que, com autorização do seu orientador e/ou com aprovação do \grupoMenor do Programa, tiver realizado o trancamento de matrícula;
		\item P: aproveitamento de créditos, atribuído ao aluno que tenha obtido aproveitamento de créditos realizados em outro Programa.
	\end{itemize}

	\begin{enumerate}
		\item Será considerado aprovado na Disciplina e terá direito a Crédito o aluno que obtiver um conceito A, B ou C.
		\item Será reprovado sem direito a Crédito o aluno que obtiver o conceito D.
	\end{enumerate}

	\item A avaliação do aproveitamento de cada Aluno será representada pelo seu Coeficiente de Rendimento, calculado semestralmente por meio de média ponderada (coeficiente de rendimento), tomando-se como peso o número de créditos das disciplinas e atribuindo-se aos conceitos A, B, C, D os valores 4,0; 3,0; 2,0; e 0,0, respectivamente.

	\begin{enumerate}
		\item O conceito D será computado para cálculo do coeficiente de rendimento enquanto outro conceito não for atribuído à disciplina repetida.
		\item As disciplinas com conceito I, S, N, J, T ou P, bem como disciplinas aproveitadas sem equivalência, não serão consideradas no cômputo do coeficiente de rendimento.
	\end{enumerate}

	\item Estará automaticamente desligado do Programa o aluno que se enquadrar em uma ou mais das seguintes situações:
	\begin{enumerate}[label=\Roman*]
		\item Obtiver coeficiente de rendimento inferior a 2,0 no seu primeiro período letivo;
		\item Obtiver coeficiente de rendimento acumulado inferior a 2,5 no seu segundo período letivo e subsequentes;
		\item Obtiver coeficiente de rendimento acumulado inferior a 3,0 no seu terceiro período letivo e subsequentes;
		\item Obtiver conceito D em disciplina repetida;
		\item Não completar todos os requisitos do curso no prazo estabelecido;
		\item Não solicitar renovação do trancamento de matrícula, quando for o caso;
		\item Não atender outras exigências estabelecidas pelo Programa em seu Regimento.
	\end{enumerate}

	\item  É obrigatória a frequência do aluno a pelo menos 75\% das atividades da Disciplina.
	\begin{enumerate}
		\singleitem Receberá conceito D na Disciplina o aluno que não estiver presente em mais de 25\% dos encontros.
	\end{enumerate}

	\item \label{conhecimentos-basicos} O Aluno para concluir seu curso deve demonstrar conhecimento em Tópicos Básicos de formação em Computação.
	\begin{enumerate}
		\item Os Tópicos Básicos são definidas pelo \grupoMenor em regulamentação própria.
		\item O Programa oferecerá regularmente disciplinas que cubram os Tópicos Básicos, denominadas Disciplinas de Formação Básica.
		\item O Programa oferecerá regularmente Provas de Proficiência nos Tópícos Básicos.
		\item \label{demonstracao-conhecimento} A demonstração do conhecimento pode ocorrer:

		\begin{enumerate}[label=\Roman*]
			\item Por aprovações em pelo menos duas Disciplinas de Formação Básica;
			\item Por aprovações em Provas de Proficiência em pelo menos dois Tópícos Básicos; 
			\item \label{isentos-conhecimento} Por ser advindo de graduação reconhecida em Ciência da Computação ou Engenharia de Computação ou de curso de mestrado acadêmico ou doutorado em Ciência da Computação ou equivalente.
		\end{enumerate}

		\item A aprovação em Prova de Proficiência não gera créditos ao aluno;

		\item Não serão computados créditos, para efeito de integralização de créditos para conclusão de curso, alunos que satisfaçam os requisitos de demonstração do conhecimento, como especificado no \ref{conhecimentos-basicos}, \ref{demonstracao-conhecimento}

		
	\end{enumerate}

	\item O Aluno para concluir seu curso deve ter aprovação em Disciplinas Eletivas ou Obrigatórias, definidas pelo \grupoMenor em resolução própria e ofertadas regularmente pelo Programa.

    \item Em caso de reprovação em uma disciplina, o aluno deverá cursá-la novamente quando de sua reedição, sendo desligado do Programa em uma segunda reprovação.

    \item Em caso de reprovação em uma disciplina de Tópicos Especiais, é facultado ao Aluno substituir esta disciplina pela execução de outra. 

	\item Será exigido dos alunos proficiência em Língua Inglesa, a qual deverá obrigatoriamente ser apresentada até a quarta matrícula no Programa.
	\begin{enumerate}
		\singleitem O Exame de Proficiência (competência) deverá ser realizado por entidade reconhecida pelo \grupoMaior.
	\end{enumerate}

\end{enumerate}

\section{Do Mestrado}
\begin{enumerate}
	\item A permanência mínima e máxima dos mestrandos no Programa de Mestrado será, respectivamente, de 12 meses e 30 meses, contados a partir da data da primeira matrícula.
	\begin{enumerate}
		\item A permanência além de 24 meses implica na apresentação de novo Seminário de Andamento pelo aluno, salvo quando data e banca para a Defesa de Dissertação tenha sido aprovadas pelo \grupoMenor.	
		\item O prazo máximo estabelecido neste Artigo poderá ser prorrogado excepcionalmente por até seis meses, por recomendação do Orientador, com aprovação do \grupoMenor, caso o Mestrando tenha cumprido todos os requisitos, exceto a apresentação da Dissertação.
	\end{enumerate}

	\item A Defesa de Dissertação deve ser solicitada ao \grupoMenor com, no mínimo, 30 dias de antecedência. 
	\item A solicitação de Defesa de Mestrado é composta por:
    \begin{enumerate}[label=\Roman*]
        \item Autorização do Orientador e ciência da Comissão de Acompanhamento Discente para marcar a Defesa;
        \item Composição da Banca Examinadora e data da Defesa;
        \item Um volume da Dissertação de Mestrado.
    \end{enumerate}
	\item Estará habilitado a solicitar a Defesa de Dissertação, o Mestrando que atenda os seguintes pré-requisitos:
	\begin{enumerate}[label=\Roman*]
		\item 	Ter completado pelo menos 20 créditos;
		\item 	Ter tido sua Proposta de Dissertação de Mestrado aprovada;
		\item 	Ter sido aprovado no Seminário de Andamento de Dissertação de Mestrado;
		\item 	Ter proficiência em Língua Inglesa comprovada.
	\end{enumerate}

	\item O Aluno que, tendo sido aprovado pela banca examinadora na defesa de Dissertação de Mestrado e cumprido os demais requisitos especificados neste Regimento, estará habilitado a receber o grau de Mestre em Ciência da Computação.

\end{enumerate}

\section{Do Doutorado}
\begin{enumerate}
	\item  A permanência mínima e máxima dos doutorandos no curso de Doutorado será, respectivamente, de 24 meses e 54 meses, contados a partir da data da primeira matrícula.
	\begin{enumerate}
		\item A permanência além de 48 meses implica na apresentação de novo Seminário de Andamento pelo aluno, salvo quando data e banca para a Defesa de Tese tenham sido aprovadas pelo \grupoMenor.	
		\item O prazo máximo estabelecido neste Artigo poderá ser prorrogado excepcionalmente por até seis meses, por recomendação do Orientador, com aprovação do \grupoMenor, caso o Doutorando tenha cumprido todos os requisitos, exceto a apresentação da Tese.
	\end{enumerate}
	
	\item A Defesa de Tese deve ser solicitada ao \grupoMenor com, no mínimo, 30 dias de antecedência. 
	\item A solicitação de Defesa de Tese é composta por:
	\begin{enumerate}[label=\Roman*]
        \item Autorização do Orientador e ciência da Comissão de Acompanhamento Discente para marcar a Defesa;
        \item Composição da Banca Examinadora e data da Defesa;
        \item Um volume da Tese de Doutorado.
    \end{enumerate}
	\item Estará habilitado a solicitar a Defesa de Tese, o Doutorando que atenda os seguintes pré-requisitos:
	\begin{enumerate}[label=\Roman*]
		\item	Ter completado pelo menos 40 créditos;
		\item	Ter tido uma Proposta de Tese de Doutorado aprovada;
		\item	Ter sido aprovado em um Seminário de Andamento;
		\item	Ter sido aprovado em Exame de Qualificação;
		\item 	Ter proficiência em Língua Inglesa comprovada;
		\item	Ter produção científica no tema da Tese de Doutorado, desenvolvida durante o Doutorado, conforme estabelecido em Resolução específica pelo \grupoMenor.
	\end{enumerate}

	\item  O doutorando que, tendo sido aprovado pela banca examinadora na defesa de Tese de Doutorado e cumprido os demais requisitos especificados neste Regimento, estará habilitado a receber o grau de Doutor em Ciência da Computação.
\end{enumerate}

\section{Das Atividades Obrigatórias e Complementares}
\label{atividades}
\begin{enumerate}

	\item  A entrega de Proposta de Dissertação ou Tese é Atividade Obrigatória e visa explicitar o problema a ser abordado pelo Aluno em sua pesquisa, estabelecendo os objetivos, argumentando sobre a relevância do problema à área, apresentando como a Proposta se distingue de trabalhos anteriores e relacionados, propondo uma metodologia de pesquisa a ser utilizada e apresentando um cronograma de trabalho.
	\begin{enumerate}
		\item A Proposta de Dissertação deverá ser entregue até a terceira matrícula do Aluno e será avaliada por no mínimo um relator.
		\item A Proposta de Tese deverá ser entregue até a quinta matrícula do Aluno e será avaliada por no mínimo dois relatores.
		\item Os relatores serão indicados pela Comissão de Acompanhamento Discente.
		\item Caso a Proposta seja reprovada, o Aluno deve apresentar nova Proposta no prazo especificados pelos relatores.
		\item A aprovação da Proposta de Tese dará direito a 2 (dois) créditos.
	\end{enumerate}

	\item O Seminário de Andamento é Atividade Obrigatória e visa o acompanhamento do trabalho do aluno de forma a verificar seu progresso, no contexto de sua Proposta, permitindo a identificação de problemas com antecedência e o conhecimento e discussão do trabalho pela comunidade.
	\begin{enumerate}
		\item O Seminário de Andamento de Mestrado deverá ocorrer após a aprovação da Proposta de Dissertação e até a quarta matrícula do aluno.
		\item O Seminário de Andamento de Doutorado deverá ocorrer após a aprovação da Proposta de Tese e até a sexta matrícula do aluno.
		\item O Seminário de Andamento será avaliado por Banca Examinadora, indicada pela Comissão de Acompanhamento Discente, em uma sessão pública.
		\item Caso reprovado no Seminário de Andamento, o Aluno deve apresentar novo Seminário no prazo especificado pela Banca Examinadora.
		\item A reprovação em dois Seminários de Andamento ou a não apresentação no prazo estabelecido levará ao desligamento do Aluno do Programa.
	\end{enumerate}

	\item  O Exame de Qualificação é Atividade Obrigatória para alunos de Doutorado e visa avaliar conhecimentos em áreas necessárias à Tese, em prazo e formato definidos pelo \grupoMenor através de Resolução própria.
	\begin{enumerate}
		\item O Exame de Qualificação será avaliado por uma Banca de Avaliação de Exame, com membros definidos pelo \grupoMenor;
		\item Não farão parte da Banca de Avaliação de Exame do Aluno seu orientador e co-orientadores.
		\item No caso de reprovação no Exame de Qualificação, o aluno poderá prestar um único novo Exame, em período máximo estipulado pela Banca de Avaliação de Exame;
		\item A reprovação em dois Exames de Qualificação ou a não prestação do Exame no prazo estabelecido levará ao desligamento do Aluno do Programa.
	\end{enumerate}

	\item Créditos podem ser obtidos pelo aluno, de forma opcional, em Atividades Complementares.
	\begin{enumerate}
		\item Para doutorandos, até oito (8) créditos podem ser obtidos nesta modalidade.
		\item Para mestrandos, até dois (2) créditos podem ser obtidos nesta modalidade.
		\item As atividades consideradas complementares, os créditos associados e as formas de verificação para cada atividade serão definidos pelo \grupoMenor em resolução própria.
	\end{enumerate}

	\item O Estágio Docência é Atividade Complementar e visa iniciar ou complementar a formação docente do Aluno, por meio da inserção deste em atividades relevantes a esta dimensão formativa.
	\begin{enumerate}
		\item Esta atividade é obrigatória para bolsistas de Doutorado, devendo estes cumprir 2 semestres de estágio.
		\item A execução será realizada no contexto de disciplinas de graduação, sob orientação e supervisão do Professor Regente da disciplina.
		\item A execução deve se distribuir ao longo de 1 semestre letivo, não podendo ultrapassar, de forma regular, 2 horas semanais.
		\item A conclusão satisfatória de 1 semestre de Estágio Docência, atestada pelo professor responsável, dará direito a 2 créditos ao aluno.
		\item Doutorandos poderão realizar o Estágio Docência por no máximo 2 semestres, enquanto Mestrandos por no máximo 1 semestre.
	\end{enumerate}	

\end{enumerate}


\section{Da Defesa de Dissertação ou Tese}

\begin{enumerate}
	\item Defesas de Dissertação ou Tese visam apresentar o trabalho realizado para a comunidade e a avaliação deste trabalho perante Banca Examinadora.
	\begin{enumerate}
		\item A Banca Examinadora para Dissertações será constituída pelo Orientador do Aluno, ou um de seus coorientadores, e por, pelo menos, um (1) avaliador externo ao Programa e um (1) avaliador pertencente ao Programa, ambos necessariamente portadores do título de doutorado.

		\item A Banca Examinadora para Teses será constituída  pelo Orientador do Aluno, ou um de seus coorientadores, e por, pelo menos, dois (2) avaliadores externos ao Programa e um (1) avaliador pertencente ao Programa, todos necessariamente portadores do título de doutorado.

		\item Ao final da Defesa, a Banca Examinadora preencherá uma Ata de Defesa, onde constará o parecer final sobre o conceito atribuído à Dissertação ou Tese apresentada e as solicitações de correções necessárias para homologação final do documento.

		\item Em casos excepcionais, quando há interesse em proteger o conhecimento gerado em função de pedido de patente, a Defesa poderá ser de caráter sigiloso, desde que solicitado pelo Orientador e seu Orientando e recebida aprovação do \grupoMenor.

		\item Na ausência do Orientador ou Coorientador para presidir a Banca, cabe ao Coordenador indicar um Docente do Programa para presidi-la.

		\item É vedado ao Presidente da Banca Examinadora emitir parecer sobre o trabalho apresentado.
	\end{enumerate}

	\item Compete ao \grupoMenor  homologar a decisão da Banca Examinadora, após parecer do Orientador sobre o atendimento das correções solicitadas na Ata de Defesa.
	\begin{enumerate}
		\singleitem A Ata de Defesa deverá conter as alterações obrigatórias a serem feitas na Dissertação ou Tese, bem como o prazo para a realização das mesmas, e as assinaturas de todos os membros da Banca Examinadora.
	\end{enumerate}

	\item Após a Defesa, e dentro dos prazos especificados na Ata de Defesa, o aluno deverá encaminhar à Secretaria do Programa, para homologação, um exemplar impresso da Dissertação ou Tese corrigida e uma cópia digital em CD ou DVD. O material entregue deverá ser acompanhado de aprovação por escrito do Orientador ou do membro indicado da Banca Examinadora na própria Ata de Defesa, conforme o caso.

	\item A redação e formatação de Dissertações e Teses deverão observar as normas estabelecidas pela Universidade Federal de Pelotas.

\end{enumerate}


\section{Da Alteração de Nível Mestrado para Doutorado}
\begin{enumerate}
	\item  A alteração do nível de mestrado para o de doutorado será permitida a alunos que contemplem os seguintes requisitos:
	\begin{enumerate}[label=\Roman*]
		\item Ter cursado no mínimo dois semestres no Programa;

		\item Apresentar Coeficiente de Rendimento igual ou superior a 3,5;

		\item Apresentar solicitação de alteração na inscrição em formulário próprio dentro do calendário do Programa, preenchido pelo Orientador.

		\item Apresentar justificativa para a mudança de nível, incluindo relatório de atividades do período em que está no mestrado e projeto para o doutorado.
	\end{enumerate}

	\begin{enumerate}
		\singleitem O \grupoMenor indicará uma comissão que avaliará o mérito da solicitação. Em caso de aprovação da solicitação, o Aluno terá um prazo de 90 dias para defender a dissertação. Tendo sua dissertação aprovada, a alteração de nível será concedida.
	\end{enumerate}

	\item Em casos especiais, a critério do \grupoMenor, durante a realização do Mestrado em Ciência da Computação será permitida a alteração da inscrição de um aluno para Doutorado, com o aproveitamento integral dos créditos já obtidos, sem a atribuição do grau de Mestre.
\end{enumerate}

\section{Das Disposições Gerais e Transitórias}

\begin{enumerate}
	\item As decisões \textit{ad referendum} devem ser submetidas à homologação do \grupoMaior ou \grupoMenor em reunião subseqüente, obedecidos seus prazos normais de ocorrência.

	\item Os casos omissos neste Regimento serão resolvidos pelo \grupoMaior, respeitando o Regimento Geral dos Cursos de Pós-Graduação Stricto Sensu.
	\begin{enumerate}
		\singleitem O Regimento Geral de Cursos de Pós-Graduação Stricto Sensu e ao Regimento Geral da Pró-Reitoria de Pesquisa e Pós-Graduação devem ser consultados para casos omissos ao presente Regimento. 
	\end{enumerate}

	\item  O presente regimento passará a vigorar a partir de sua aprovação pelo Conselho Coordenador de Ensino, Pesquisa e Extensão desta Instituição.

\end{enumerate}

\end{document}